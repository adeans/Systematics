\documentclass[letterpaper, 11pt]{article}
\usepackage{comment} % enables the use of multi-line comments (\ifx \fi) 
\usepackage{fullpage} % changes the margin
\usepackage{fancyhdr} % for footer
\usepackage[UKenglish]{isodate}% http://ctan.org/pkg/isodate for date format
\usepackage[letterpaper,margin=1in]{geometry}
\usepackage{changepage}%for dichotomous key
\usepackage{float}%force tables/figs into certain placement
\usepackage{natbib}	%for bibliography
\usepackage{placeins}%prevent images from floating into inappropriate sections
\usepackage{subcaption}%for figures
\usepackage{hyperref}%for links
\newcommand*{\doi}[1]{\href{http://dx.doi.org/#1}{doi: #1}}% links DOI
%\usepackage{epigrafica}%changes default font to epigrafica
\newcommand{\latinword}[1]{\texttt{\itshape #1}}%use \latinword for vocab
\newenvironment{labelfontsmall}{\fontfamily{phv}\selectfont}{\par}%for label font example
\def\labelitemi{--} %change bullet to em dash

\pagestyle{fancy}
\renewcommand{\headrulewidth}{0pt}

\lhead{}
\chead{}
\rhead{}
\lfoot{Systematics, past and present}
\cfoot{}
\rfoot{\thepage}
\renewcommand{\footrulewidth}{0.4pt}
\title{Systematics, past and present}
\author{Andrew R. Deans}
\begin{document}
\cleanlookdateon %removed ordinal date
\maketitle
\thispagestyle{fancy}
\section*{Introduction}
Systematists are charged with describing, naming, estimating evolutionary history, and classifying organisms. Here we cover the history of classification, so that you can understand the context of systematics today. \cite{grimaldi2005evolution} and \cite{EngelKristensen2013} together provide a thorough review of this history if you're interested in more details.

In this unit, we also remind ourselves of certain core concepts in evolution, which will facilitate our understanding of how Arthropoda radiated and evolved. We'll also discuss the kinds of data and analytical approaches (and even philosophies) we use to estimate evolutionary history (\textit{i.e}., phylogeny).

\section{History, concepts, and people to know}
We'll open this unit with a discussion and slideshow about the history of systematics, as it relates to arthropods, and a reintroduction to evolutionary concepts. Takes notes and think about the following questions before moving on to lab-oriented exercises. You don't need to answer these questions in your lab notebook (You can if you want!), but you may be tested at some point on these concepts and big picture questions.
\subsection{Systematists}
If asked to describe the accomplishments of these people in 1--2 sentences could you do it? Could you arrange these figures chronologically?

\begin{enumerate} 
\item{Aristotle} 
\item{Charles Darwin}  
\item {Johann C. Fabricius}
\item {Willi Hennig}
\item Niels P. Kristensen
\item{Carolus Linn\ae{us}}  
\item{Maria Sibylla Merian}  
\end{enumerate}

\subsection{Concepts}
Think about the concepts and questions below. Could you write a couple sentences that explain each term? Can you provide examples? Cycle through the hands-on exercises and come back to this section later. 

\begin{enumerate} 
\item classification
\item nomenclature
\item phylogenetics 
\item{natural selection} 
\item{adaptation}  
\item{species}  
\item {adaptive radiation}
\item {monophyletic \textit{vs}. polyphyletic}  
\item {homology \textit{vs}. homoplasy}  
\item {ancestral/plesiomorphic \textit{vs}. derived/apomorphic}
\end{enumerate}

\noindent{}Can you draw a phylogeny and interpret its meaning? Why are these diagrams important? \cite{baum2008phylogenics} provide a helpful guide.\\

\noindent{}Can you describe three technologies or tools that allowed for major leaps forward in insect taxonomy? (You can probably think of more than three.)\\

\noindent{}We discussed three types of data that are typically used to estimate phylogenies: phenotype (including behavior), ecology, and molecules (DNA, proteins, \textit{etc}.) Can you briefly describe some advantages and limitations of each data type?\\

\noindent{}Can you describe the differences between a phenetic and a cladistic approach to phylogeny? What is our current approach to character and phylogeny interpretation? \\

\section{Hands-on systematics exercises}
As you learned in Unit 2, the external features of an arthropod, \textit{i.e}., the phenotypes we can easily observe through a microscope, are the primary source of characters we use to diagnose arthropods and to hypothesize evolutionary relationships. The goal of this lab is to review the morphological concepts you learned in the last unit and then to develop characters (descriptions of phenotypic variation) that help you, an inspired systematist, understand and communicate about Arthropoda.

\subsection*{Specimens}

\begin{enumerate}
	\item {spider (Araneae)}
	\item {harvestman (Opiliones)}
	\item {woodlouse (Isopoda)}
%	\item {mayfly (Ephemeroptera)}
	\item {grasshopper or katydid (Orthoptera)}
    \item mole cricket (Orthoptera)
	\item {backswimmer or related (Heteroptera: Notonectidae)}
	\item {plant/leaf hopper(Hemiptera)}
%	\item {ground beetle (Coleoptera: Carabidae)}
%	\item {ground beetle bleached with H\textsubscript{2}O\textsubscript{2} (Coleoptera: Carabidae)}
	\item {weevil (Coleoptera: Curculionidae)}
    \item Japanese beetle or related (Coleoptera)
	\item {moth (Lepidoptera)}
	\item {robber fly (Diptera: Asilidae)}
	\item {fly (Diptera)}
	\item {wasp (Hymenoptera)}
	\item {sawfly (Hymenoptera)}
%	\item {springtail (Collembola)}
	\item {antlion (Neuroptera: Myrmeleontidae)}
    \item caterpillar (Lepidoptera)
\end{enumerate}

\section{Classify your specimens} % move this to unit 3
Take some time to sort the specimens into groups. \\

\hangindent2em{}\textbf{Question 1:} How many groups did you make, and what body parts provided the evidence or characters to justify your groups?

\subsection{Anatomy refresher (10--15 min)}
Look at your lab notebook, and review your answers to the Unit 2 questions. Let's see if we can write formal  definitions for some of the concepts we learned. Here is the typical structure of a \textit{genus differentia}: A \textit{species} is a \textit{genus} that (some description of how it differs from other species in that genus). Here is an example: A coffee is a hot beverage that is brewed from the seeds of \textit{Arabica} plants. ``Coffee'' is the species here, and it belongs in the broader class (the Genus) ``hot beverage''. Coffee differs from other hot beverages by being the one that was brewed from the seeds of \textit{Arabica} plants (\textit{i.e}., this is the \textit{differentia}).\\

\hangindent2em\textbf{Question 2:} Can you generate a \textit{genus differentia} definition for this anatomical concept: \\

A \latinword{head} is a \makebox[30 mm]{\hrulefill} that \hrulefill.\\ \\

\hangindent2em{}What about for \latinword{ocellus}, \latinword{antenna}, \latinword{pedicel}, \latinword{mandible}, \latinword{maxilla}, \latinword{leg}, \latinword{fore leg}, \latinword{wing}, \latinword{fore wing}, \latinword{mesothorax}? \\

\subsection{Character and matrix development (30 min)}
Think of a \textbf{character} as a relatively broad class of possible phenotypes for a body part. ``Flagellomere shape'', ``femur color'', and ``claw presence'' are three examples of characters. Character states are more specific versions of the character, and they're typically numbered arbitrarily from 0 to $\infty$:\\

Flagellomere shape:	0 = flagellum round, 1 = flagellum square\\

Femur color: 0 = femur red, 1 = femur orange, 2 = femur black\\

Claw presence: 0 = present, 1 = absent\\

With this list of characters, and a list of the taxa you're examining, we can score a matrix:
\begin{table}[H]
\centering
\label{my-label}
\begin{tabular}{lccc}
                             & \multicolumn{1}{l}{flagellomere shape} & \multicolumn{1}{l}{femur color} & \multicolumn{1}{l}{claw presence} \\ \cline{2-4} 
\multicolumn{1}{l|}{insect A} & \multicolumn{1}{c|}{0}                 & \multicolumn{1}{c|}{0}          & \multicolumn{1}{c|}{1}            \\ \cline{2-4} 
\multicolumn{1}{l|}{insect B} & \multicolumn{1}{c|}{1}                 & \multicolumn{1}{c|}{2}          & \multicolumn{1}{c|}{1}            \\ \cline{2-4} 
\multicolumn{1}{l|}{insect C} & \multicolumn{1}{c|}{0}                 & \multicolumn{1}{c|}{1}          & \multicolumn{1}{c|}{0}            \\ \cline{2-4} 
\end{tabular}
\end{table}

\noindent{}We can compute across matrices like this using software that provides us with phylogenetic trees and software that generates diagnostic tools (\textit{e.g}., multi-entry keys).\\

\hangindent2em\textbf{Question 3:} Choose eight of your specimens, and see how many characters you can create for them. Mimic the descriptive style you see in the examples above, and score your taxa in a matrix. How do you account for homology?\\

\subsection{Dichotomous keys (30 min)}
Dichotomous keys are common diagnostic tools, and you will almost certainly use several in your efforts to determine specimens in your collection. They are composed from character sets, like the table you generated above and typically look like this (adapted from \cite[][page 151]{borror1989introduction}:\\

\begin{adjustwidth}{1cm}{}
1. With well-developed wings (adults) \dotfill{} 2\\
1$'$. Wingless or with wings vestigial or rudimentary (nymphs, larvae, and some adults) \dotfill{} 30\\

\noindent{}2(1). Wings membranous, not hardened or leathery \dotfill{} 3\\
2$'$. Front wings hardened or leathery, at least at base; hind wings, if present, usually membranous \dotfill{}24\\
\end{adjustwidth}

\noindent{}Think of it almost as a series of if-then statements. If your specimen has well-developed wings then go to line 2. If those wings are membranous then go to line 3.\\

\noindent{}Look through the keys provided on the workbench. See any patterns? Try running a few of your specimens through the key. \\

\hangindent2em\textbf{Question 4:} Write a dichotomous key using the characters you developed for Question 2. What kinds of characters would you use first, those that are easy to understand or those that separate the taxa into similarly-sized groups?\\

\subsection{Thinking phylogenetically (30 min)}
Look at your list of characters and ask yourself how many provide evidence concerning the evolution of these organisms. Which states are ancestral and which ones are derived, and why? How many do you think are purely diagnostic, offering no usable information for understanding evolution?\\

\hangindent2em\textbf{Question 5:} Draw a phylogenetic tree that represents your hypothesis for the evolutionary history of your specimens. Your instructors will demonstrate how to do this on the board. Map your character states on the tree. Do your data support your tree? Can you see evidence of convergent evolution?\\

\subsection{Solving nomenclatural problems (30 min)}

We'll discuss the \cite{iczn} and its Code, portions of which are in the appendix of this handout. Read the following descriptions of nomenclatural problems, and see if you can find solutions in the pieces of the Code given to you in this handout (see Appendix I). \\

\hangindent2em\textbf{Question 6:} Linn\ae{}us described the honey bee in 1758, naming it \textit{Apis mellifera}. Joe Schmoe insists on calling these insects \textit{Honeyus beeus}, however, because he found this name in an even older volume (1725), by another author. Joe thinks \textit{Honeyus beeus} should have priority, as spelled out in Article 23. What is the correct name? Which article resolves this conundrum?\\

\hangindent2em\textbf{Question 7:} Jane Doe tries to publish a new species name with a modern twist---a crane fly named \textit{Tipula p0k3m0norum} Doe, 2016. The peer review process informs her that this is not possible. Which article do her peers reference in their critique and why?\\

\hangindent2em\textbf{Question 8:} Frustrated with the high expense and slow rate of scientific publication (and the constant rejection of manuscripts by his peers), Randy Bobandy decides to self-publish all new species. He prints his manuscripts in his mom's basement, avoids peer-review, and mails them to several local libraries to be archived. Are these new names available? Why or why not?\\

\section{Specimen data and accesibility (45 min)}

The next set of exercises introduce you to some basic \latinword{biodiversity informatics}. You've started collecting and sorting insects, undoubtedly from numerous localities. You've also made specimen preparations and labeled each one according to collection best practices. As part of this process you've also established a database of specimens in your collection: a spreadsheet with columns for locality, collector, identifier, \textit{etc}. You should be using the prescribed spreadsheet format for this class, but Appendix II describes how to make non-standard spreadsheets accessible.\\

\noindent{}We'll talk a bit about biodiversity data standards and tools and other resources that facilitate data sharing. 

\section*{Glossary}
We'll be using terms you may not be familiar with. Some of these concepts are provided below, and we'll go over them together. If you have a new one to add raise your hand!
\begin{enumerate} 
\item {Darwin Core}
\item {GBIF}
\item {IPT}
\item {metadata}
\item{TDWG} 
\item{XML}  
\end{enumerate}

\section*{Your data}
Take a look at this label, which probably looks similar to the labels you've created for your specimens:\\

\begin{labelfontsmall}
\tiny
\noindent{USA: PA: Centre County: \\ Pine Grove Mills, 40.730, \\ -77.884, $\pm$ 250m 15.iv.2016 \\ A.R. Deans, sifted litter}
\end{labelfontsmall}
\normalsize\vspace{5mm}

\hangindent2em\textbf{Question 9:} What kinds of data do you see represented? How many kinds are there? List them.\\

\hangindent2em\textbf{Question 10:} If you had these data from millions of specimens---all 35+ million insects in the Smithsonian Institution, for example---what kinds of hypotheses could you test? See if you can think of three example research questions.\\

%\hangindent2em\textbf{Question 11:} Now look at your spreadsheet (not the one for this class, but one that was generated for an outside project). How many columns do you have? Is each one a data type? How would you enter the label above into your spreadsheet? What data type(s) is/are in your spreadsheet but not represented in that label above? \\

%\hangindent2em\textbf{Question 12:} Compare your outside project spreadsheet to your neighbors' outside spreadsheets. How are they different?

\section*{Biodiversity data in research}
Biodiversity science is incredibly rich, with respect to the array of research questions and the data types that can be applied to them. We'll discuss a few examples of research that relies on collections data. As we talk, think about the minimum data required for these kinds of questions.

\section*{Biodiversity data standards}
Hopefully you thought of some compelling research questions for Question 10. How can we aggregate data from hundreds (or thousands!) of natural history collections to test your hypotheses? Your spreadsheet almost definitely differed from your neighbors' databases, and you can imagine that a similar scenario exists in the natural history collections world. The Smithsonian would use a different approach than the American Museum of Natural History and the Field Museum. \\

\noindent{}Fortunately there are established biodiversity data standards and tools that facilitate sharing. We'll look at those developed by TDWG, especially the Darwin Core and associated resources. Your homework: Open a Web browser and navigate to \url{http://rs.tdwg.org/dwc/terms/index.htm}. This massive list almost definitely includes the data types you identified in Question 9, along with dozens more that may or may not be relevant to your collection.\\

\noindent{}Find the following terms and read their definitions: \latinword{catalogNumber}, \latinword{recordedBy}, \latinword{eventDate},\\ \latinword{samplingProtocol}, \latinword{fieldNotes}, \latinword{higherGeography} and \latinword{locality}, \latinword{decimalLatitude},\\ \latinword{decimalLongitude}, \latinword{coordinateUncertaintyInMeters}, \latinword{scientificName}. Do any of these sound familiar?\\

\hangindent2em\textbf{Question 11:} How many of your spreadsheet column headers are represented in the list above? Based on your reading of these (and maybe your eyes strayed to other terms), how would you change your spreadsheet organization? Or would you?

\section*{More thoughts on biodiversity data}
Now that you have an idea of how to create a set of files that contribute to the greater scientific enterprise it's time to think about how to extend and enrich your data sets. We'll discuss some of these issues as a group.
\begin{itemize}
\item Many journals (\textit{e.g.}, PLoS) and granting agencies want information about specimen provenance and collecting permits. How would you incorporate that into your spreadsheet or archive?
\item All of your specimens will be deposited at the Frost Entomological Museum. How do we specify that in your file?
\item Do any of you have images or videos? How do we associate them with specimen records?
\item How do we associate specimens, for example a parasitoid and its host? 
\item How do we account for a range of dates, as we might see with a Malaise trap?
\item How to explain our approach to georeferencing (\textit{i.e.}, finding a latitude and longitude for each specimen)? Do you know what geodetic datum means?
\item Can biodiversity data be copyrighted?
\end{itemize}

\section*{\textit{ProTip}\texttrademark}
GBIF provides structured spreadsheets as Microsoft Excel files: 
\url{https://github.com/gbif/ipt/wiki/occurrenceData#templates}. These files can be imported through any of the myriad Integrated Publishing Toolkit (IPT) instances.

\clearpage
\section*{Epilogue}
This handout is part of an open curriculum, initially developed by Andrew R. Deans at the Pennsylvania State University. Original files are available free for anyone to download, copy, modify, and improve at the Open Entomology GitHub repository \citep{ENT532}, which also provides a mechanism for reporting problems and other feedback:\\
\url{https://github.com/OpenEntomology/InsectBiodiversityEvolution/issues}


\FloatBarrier
% adding bibliography here
\bibliographystyle{myplainnat}
\bibliography{bib}
\clearpage

\section*{Appendix I. Portions of the International Code of Zoological Nomenclature}
The following text was copied from the ICZN online edition (\url{http://www.iczn.org/code}). It has been simplified for the exercises in this handout. Not all articles are included.\\

\noindent{}\textbf{Article 3. Starting point.} The date 1 January 1758 is arbitrarily fixed in this Code as the date of the starting point of zoological nomenclature.\\

\noindent{}\textbf{Article 5. Principle of Binominal Nomenclature.}\\
\noindent{}\textbf{5.1. Names of species.} The scientific name of a species, and not of a taxon of any other rank, is a combination of two names (a binomen), the first being the generic name and the second being the specific name. The generic name must begin with an upper-case letter and the specific name must begin with a lower-case letter.\\

\noindent{}\textbf{Article 8. What constitutes published work.} A work is to be regarded as published for the purposes of zoological nomenclature if it complies with the requirements of this Article and is not excluded by the provisions of Article 9.\\
\noindent{}\textbf{8.1. Criteria to be met.} A work must satisfy the following criteria:\\
\noindent{}\textbf{8.1.1.} it must be issued for the purpose of providing a public and permanent scientific record,\\
\noindent{}\textbf{8.1.2.} it must be obtainable, when first issued, free of charge or by purchase, and\\
\noindent{}\textbf{8.1.3.} it must have been produced in an edition containing simultaneously obtainable copies by a method that assures\\
\noindent{}\textbf{8.1.3.1.} numerous identical and durable copies, or\\
\noindent{}\textbf{8.1.3.2.} widely accessible electronic copies with fixed content and layout.\\

\noindent{}\textbf{Article 9. What does not constitute published work.} Notwithstanding the provisions of Article 8, none of the following constitutes published work within the meaning of the Code:

\noindent{}\textbf{9.1.} after 1930 handwriting reproduced in facsimile by any process;\\
\noindent{}\textbf{9.2.} after 1985, works produced by hectographing or mimeographing;\\
\noindent{}\textbf{9.3.} before 1986 and after 2012, works issued on optical discs;\\
\noindent{}\textbf{9.4.} photographs as such;\\
\noindent{}\textbf{9.5.} proof sheets;\\
\noindent{}\textbf{9.6.} microfilms;\\
\noindent{}\textbf{9.7.} acoustic records made by any method;\\
\noindent{}\textbf{9.8.} labels of specimens;\\
\noindent{}\textbf{9.9.} preliminary versions of works accessible electronically in advance of publication (see Article 21.8.3);\\
\noindent{}\textbf{9.10.} materials issued primarily to participants at meetings (e.g. symposia, colloquia, congresses, or workshops), including abstracts and texts of presentations or posters;\\
\noindent{}\textbf{9.11.} text or illustrations distributed by means of electronic signals (\textit{e.g.} via the Internet), except those fulfilling the requirements of Articles 8.1 and 8.5.\\
\noindent{}\textbf{9.12.} facsimiles or reproductions obtained on demand of an unpublished work [Art. 8], even if previously deposited in a library or other archive.\\

\noindent{}\textbf{Article 11. Requirements [for the availability of a name].} To be available, a name or, where relevant, a nomenclatural act must satisfy the following provisions:\\
\noindent{}\textbf{11.1. Publication.} The name or nomenclatural act must have been published, in the meaning of Article 8, after 1757.\\
\noindent{}\textbf{11.2. Mandatory use of Latin alphabet.} A scientific name must, when first published, have been spelled only in the 26 letters of the Latin alphabet (taken to include the letters j, k, w and y); the presence in a name when first published of diacritic and other marks, apostrophes or ligatures, or a hyphen, or a numeral in a compound species-group name, does not render the name unavailable (for corrections, see Articles 27 and 32.5.2).\\
\noindent{}\textbf{11.3. Derivation.} Providing it meets the requirements of this Chapter, a name may be a word in or derived from Latin, Greek or any other language (even one with no alphabet), or be formed from such a word. It may be an arbitrary combination of letters providing this is formed to be used as a word.\\

\noindent{}\textbf{Article 14. Anonymous authorship of names and nomenclatural acts.} A new name or nomenclatural act published after 1950 with anonymous authorship is not thereby made available.\\

\noindent{}\textbf{Article 18. Inappropriate and tautonymous names.} The availability of a name is not affected by inappropriateness or tautonymy.\\

\noindent{}\textbf{Article 23. Principle of Priority.}\\
\noindent{}\textbf{23.1. Statement of the Principle of Priority.} The valid name of a taxon is the oldest available name applied to it, unless that name has been invalidated or another name is given precedence by any provision of the Code or by any ruling of the Commission. For this reason priority applies to the validity of synonyms, to the relative precedence of homonyms, the correctness or otherwise of spellings, and to the validity of nomenclatural acts (such as acts taken under the Principle of the First Reviser and the fixation of name-bearing types).\\

\noindent{}\textbf{Article 61. Principle of Typification.}\\
\noindent{}\textbf{61.1. Statement of the Principle of Typification.} Each nominal taxon in the family, genus or species groups has actually or potentially a name-bearing type. The fixation of the name-bearing type of a nominal taxon provides the objective standard of reference for the application of the name it bears.\\

\noindent{}\textbf{Article 76. Type locality.}\\
\noindent{}\textbf{76.1. Definition.} The type locality of a nominal species-group taxon is the geographical (and, where relevant, stratigraphical) place of capture, collection or observation of the name-bearing type; if there are syntypes and no lectotype has been designated, the type locality encompasses the localities of all of them.\\

\noindent{}\textbf{Article 78. Powers and duties of the Commission.}\\
\noindent{}\textbf{78.1. Plenary Power.} The Commission is empowered, by a resolution of the IX International Congress of Zoology (1913) and ratified in subsequent Codes by its successors, under conditions specified in Article 81 to suspend the application in a particular case of any provision of the Code except those in the present and next succeeding Chapter. The course to be followed is decided by the Commission under this plenary power and its ruling is published in an Opinion [Art. 80.2].



\clearpage


\section*{Appendix II. Prepare your data!}

GBIF provides a tool, the Darwin Core Archive Assistant \citep{DWCaGuide}, that will make each of your data sets (spreadsheets)---no matter how different---available more broadly for research. We won't do this in class, but if you want to use this approach to sharing old spreadsheets here's how.

\section*{Methods and materials}
In the past we worked at computers for these exercises, but this year we will mostly just discuss the methods. The original exercise follows, which you can try at home, on your own. Software and datasets required:
\begin{itemize}
\item Web browser (\textit{e.g.}, Firefox or Chrome)
\item Text editor (TextWrangler for OSX or Notepad++ for Windows)
\item Spreadsheet editor (\textit{e.g.}, Google Sheets, Microsoft Excel)
\item Your data!
\end{itemize}

\begin{enumerate}
\item Open a Web browser and navigate to \url{http://tools.gbif.org/dwca-assistant/}.
\item Our data are primarily ``occurrences'' (which specimens occurred where), so check that box in the upper left, under ``Core''.
\item On the upper right of the page you'll see a place to type the name of your spreadsheet file (``Filename:''); make sure it ends with .csv, as we'll be exporting your data as a \textbf{c}omma \textbf{s}eparated \textbf{v}alues file.
\item When we selected ``occurrences'' the tool auto-populated our list with two required terms (fields), \latinword{ID} and \latinword{basisOfRecord}. Open a new tab in your browser and navigate back to \url{http://rs.tdwg.org/dwc/terms/index.htm}; find and read about \latinword{basisOfRecord}. We'll discuss \latinword{ID} as a group.
\item Now go back to the Darwin Core Archive Assistant. Find the terms under Occurrences (far left) that match the columns in your spreadsheet. Check each one that applies.
\item The terms should have been added to the area in the middle. Are they in the same order as your spreadsheet columns? Item ``0'' in the list of terms should match column ``A'' in your spreadsheet, item 1 to column B, \textit{etc}. Note that you can drag the terms up and down in that list. You can also insert a ``spacer'' for columns in your spreadsheet that do not match any Darwin Core terms. I recommend dragging \latinword{ID} to the bottom of your list.
\item Once you have a list of terms that matches your spreadsheet headers it's time to validate! Navigate to the tab that reads ``meta.xml''. You should see an XML file that describes your database. Click ``Save File'' and export the file to your desktop. You should also export your spreadsheet as a CSV file. Put both of these files---the CSV and XML files---inside a folder. Now compress that folder into a .zip
\item Go back to the ``meta.xml'' tab and click ``Validate''. This should launch the validator in a new browser tab. Find ``Upload local archive:'', choose your zipped folder and click ``Validate''. Did you get any warnings? Look at the rendering of your data at the bottom of the page. Do the columns line up? We'll discuss the results and any questions as a group.
\end{enumerate}

\end{document}


%%%%%%%%%%% GARAGE

\subsection*{Materials}

\begin{itemize}
\item fine forceps, probes (provided)
\item sorting tray, watch glasses, gloves, safety glasses, ethanol (provided)
\item {lab notebook (\textit{not} provided!)}
\end{itemize}

\section*{Safety}
We will be working with sharp tools. Wear your personal protective gear at all times. Specimens are to be returned to their vials after lab, and glycerol will be collected for proper disposal or reuse.

\section*{Methods}
Working with a partner, organize your space, specimens, tools, and microscope. Use your probe and forceps to manipulate the specimen, and don't be afraid to pull parts off or otherwise dissect the specimens. Use the Unit 2 morphology handout as a guide.